%!TEX root = ../maturaarbeit.tex

\section{Grundlagen}

\subsection{Aufbau von Galaxien}

Es gibt verschiedene Arten von Galaxien wie zum Beispiel die Eliptische-, Spiral-, Balken-, Linsenförmige-, 
und die Irreguläre Galxien. Sie unterscheiden sich in Struktur und Form. Grundsätzlich haben alle ein 
Gravitationszentrum, um welches sich alles dreht. Im Zentrum der Galaxie befindet sich ein Schwarzes 
Loch welches umgeben von einem sogenannten Bulge ist. Der Bulge ist eine riesige Sternenansamung auf relativ engem Raum. Um das 
Zentrum herum erstreckt sich eine galaktische scheibe, in der sich die meisten Sterne, interstellarem Gas und Staub befinden. 
Je nach Galaxieart sind die "Arme" anders angeordnet, z.B. sind diese spriralförmig in der Spiralgalaxie. Um die Scheibe herum befindet 
sich ein spärisches Halo, in dem sich Sterne und Kugelsternhaufen befinden. 
(Ignasi Ribas, Chris Hadfield, 2021, S. 20-24)

\subsection{Galxien drehen sich zu schnell}
Die Formel zu Berechnung von der Orbital Geschwindigkeit lautet:
\begin{align*}    
    v = \sqrt[2]{\frac{G * M}{r}}
\end{align*}
Wobei G die Gravitstionskonstante ist, M ist die Masse des Zentralgestirns und r die distanz zwischen den Objekten. Bei M muss 
beachtet werden, das die gesamte Masse innerhalb der jeweiligen Planetenbahn zählt. 
Von dieser Formel würde man darauf schliessen, je weiter weg das Objekt desto lamgsamer bewegt sich 
dieses, da durch r dividiert wird, was auch bei den Planeten in unserem Sonnensystem zu trifft. Dieses Gesetz kann auf alle
Systeme wo ein Körper ein Zentralbereich umkreist angewendet werden.  (Thomas Bührke, 2022, S. 21)
\subsection{Rotationskurve}
Rotationskurven geben an, wie sich die Geschwindigkeit von Objekten innerhalb einer Galaxie verändern im Vergleich zum Abstand vom Zentrum. 
Damit die Rotationskurve und die Massenverteilung von Galaxien berechnet werden kann, muss ihre Struktur sowie ihre Zusammensetzung 
aus Sternen, Staub, stellarem Gas und dunkler Materie bekannt sein. (Bachelorarbeit, Ann-Kristin Möller, S.5) 

Die universale Rotationskurve, ist eine mathematische Formel, die die Rotationskurven von Galaxien anhand ihrer Gesamthelligkeit und ihres Radius charakterisiert.
Wodurch darauf geschlossen werden kann, dass die hellsten Galaxien eine leicht Abfallende Rotationskurve haben, mittlere eine constant flache und dunklere 
Galaxien eine monoton ansteigende Rotationskurve haben. (Yoshiaki Sofue and Vera Rubin, S.7 und s 17 )
\subsection{Die vier fundamentale Wechselwirkungen}
Die vier Wechselwirkungen sind, die Gravitation, die elektromagnetische Wechselwirkung, die schwache Wchselwirkung und die 
starke Wechselwirkung. Jede Wechselwirkung hat eine eigene Ladung, diese gibt an wie sensitiv ein Teilchen für diese Wechselwirkung ist. 
Es könnten alle Phänomene und Prozesse, welche auf der Erde oder im Weltall beobachtet werden mit den vier fundamentalen Wechselwirkungen beschrieben werden.
Das Standardmodell der Teilchenphysic beschreibt drei der Vier wechselwirkungen, die Gravitation spielt keine zentrale Rolle, da die Teilchen eine solch kleine Masse haben.
Die Gravitation wird durch die allgemeine Relativitätstheorie beschriebn die nicht Teil des Standardmodels ist. Sie wirkt zwischen allen Teilchen welche eine Masse besitzen. Die elektromagnetische
Wechselwirkung , wirkt zwischen elektricsch geladenen Teilchen wie Elektronen und Protonen. Die schache wechselwirkung wirkt bei der Kernfusion. Die starke Wechselwirkung hält die Protonen zusammen(Thomas Bührke, S.181 + website)
Die Gravitation und die elektromagnetische Wechselwirkung erfahren wir die direkt in unserem ALltag, die schache und starke wechselwirkung jedoch nicht da ihre Reichweite zu gering ist.
\subsection{Dunkle Materie} 
Es gibt zwei Arten von Dunkler Materie. Die baryonische und die nicht baryonische Dunkle
Materie. Unter baryonischer Materie versteht man "normale" Materie welche aus Elektronen, Neutronen 
und Protonen besteht. Damit sind zum Beispiel Objekte wie Massenarme und daher Leuchtschwache Sterne
gemeint. Diese haben die Fachbezeichnung "MACHO" was für MAssive Compact Halo Objects steht. Diese können
nicht Direkt beobachtet werden. Braune Zwerge, Weisse Zwerge und Neutronensterne sowie schwarze Löcher gehören auch zu der Kategorie 
baryonische dunkle Materie. (S.46 Thomas Bührke) Jedoch reicht die Masse dieser Objekte nicht aus um das dunkle Materie Problem zu lösen. (Bachelor 4.1) 
Nicht baryonische Materie, wie der Name verrät besteht nicht aus Elektronen, Neutronen und Protonen. Es gibt
verschiede Vermutungen was diese sein könnte. (Sibylle Anderl,2023, S. 51) 
In frage kommen WIMPs, was auf deutsch für schwach wechselwirkende massenreiche Teilchen bedeutet. Diese Teilchen besitzen keine Ladung und somit auch kein elektrisches oder magnetisches Feld, dadurch beschränkt sich ihre 
Wechselwirkung auf die Gravitation und die schwache Wechselwirkung. Das bedeutet das sie ungestört Plaenten durchqueren wodurch sie auch so schwer sind zu dedektieren. (chemie.de)
Es wird mit verschiedenen Methoden nach WIMPs gesucht. Die Suchmethoden basieren darauf, dass ein WIMP mit einem Atomkern zusammenstosst dies kann unterschiedliche Folgen haben, wie die Ionisation, die Szintillation oder die Phonen-Anregung.
Bei der Ionisation schlägt das WIMP ein Elektron aus der Atomhülle heraus und ein messbaren elektrischen Strom wird erzeugt. Bei der Szintillation löst das WIMP einen kurzen Lichtblitz aus,
wenn es auf ein Atomkern prallt. Bei der Phonen-Anregung kann die Kolision mit einem WIMP in einem Kristall Schiwingungen des Kristallgitters auslösen und dabei erhöht sich die Temperatur geringfügig. (Thomas Bührke S.190)

(Dr.Thomas Bührke, S. 188)
\subsection{Erstes Unterkapitel des ersten Kapitels}
\subsection{Zweites Unterkapitel des ersten Kapitels}
Füge mit \textit{newpage} einen Seitenumbruch ein.

