%!TEX root = ../maturaarbeit.tex


\section{Die vier fundamentalen Wechselwirkungen}

Die vier Wechselwirkungen setzen sich aus der Gravitation, der elektromagnetischen Wechselwirkung, der schwachen Wechselwirkung und der starken Wechselwirkung zusammen. Jede Wechselwirkung hat eine eigene Ladung. Es können alle Phänomene und Prozesse, welche auf der Erde oder im Weltall beobachtet werden, mit den vier fundamentalen Wechselwirkungen beschrieben werden.
Das Standardmodell der Teilchenphysik beschreibt drei der vier Wechselwirkungen, die Gravitation spielt keine zentrale Rolle, da die Teilchen eine sehr kleine Masse haben. Die Gravitation wird durch die allgemeine Relativitätstheorie beschrieben, die nicht Teil des Standardmodells ist. Sie wirkt zwischen allen Teilchen, welche eine Masse besitzen. 
Die elektromagnetische Wechselwirkung wirkt zwischen elektrisch geladenen Teilchen wie Elektronen und Protonen. Die schwache Wechselwirkung wirkt bei der Kernfusion. Die starke Wechselwirkung hält die Protonen zusammen.
Die Gravitation und die elektromagnetische Wechselwirkung erfahren wir direkt in unserem Alltag, die schwache und starke Wechselwirkung jedoch nicht, da ihre Reichweite zu gering ist \cite{Bührke2022}.

\section{Dunkle Materie}

Es gibt zwei Arten von Dunkler Materie, die baryonische und die nicht baryonische Dunkle Materie.\\ Unter baryonischer Materie versteht man normale Materie, welche aus Elektronen, Neutronen und Protonen besteht. Damit sind zum Beispiel Objekte wie massenarme und daher leuchtschwache Sterne gemeint. Diese haben die Fachbezeichnung \glqq MACHO\grqq{} was für \glqq MAssive Compact Halo Objects\grqq{} steht. Diese können nicht direkt beobachtet werden. Braune Zwerge, Weisse Zwerge und Neutronensterne sowie Schwarze Löcher gehören auch zu der Kategorie baryonische Dunkle Materie \cite{Bührke2022}.
Jedoch reicht die Masse dieser Objekte nicht aus, um alle Beobachtungen im All zu erklären\cite{Moeller2010}.\\ Nicht baryonische Materie, wie der Name verrät, besteht nicht aus Elektronen, Neutronen und Protonen. Es gibt verschiedene Vermutungen, was diese sein könnten \cite{Anderl2023}.\\
In Frage kommen WIMPs (weakly interacting massive particles), was auf Deutsch schwach wechselwirkende, massenreiche Teilchen bedeutet. Diese Teilchen besitzen keine Ladung und somit auch kein elektrisches oder magnetisches Feld. Dadurch beschränkt sich ihre Wechselwirkung auf die Gravitation und die schwache Wechselwirkung. Das heisst, dass Dunkle Materie ungestört Planeten durchqueren könnte, wodurch sie auch so schwer zu detektieren sind. 
Es wird mit verschiedenen Methoden nach WIMPs gesucht. Die Suchmethoden basieren darauf, dass ein WIMP mit einem Atomkern zusammenstösst und das kann unterschiedliche Folgen haben, wie die Ionisation, die Szintillation oder die Phononen-Anregung. Bei der Ionisation schlägt das WIMP ein Elektron aus der Atomhülle heraus und ein messbarer elektrischer Strom wird erzeugt.
Bei der Szintillation löst das WIMP einen kurzen Lichtblitz aus, wenn es auf einen Atomkern prallt. Bei der Phononen-Anregung kann die Kollision mit einem WIMP in einem Kristall Schwingungen des Kristallgitters auslösen, und dabei erhöht sich die Temperatur geringfügig. Jedoch wurde bis heute die Existenz von WIMPs noch nicht nachgewiesen. \cite{Bührke2022} 
