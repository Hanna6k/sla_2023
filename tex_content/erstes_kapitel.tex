%!TEX root = ../maturaarbeit.tex

\section{Kapitel 2}

\subsection{Aufbau von Galaxien}

Es gibt verschiedene Arten von Galaxien wie zum Beispiel die Eliptische-, Spiral-, Balken-, Linsenförmige-, 
und die Irreguläre Galxien. Sie unterscheiden sich in Struktur und Form. Grundsätzlich haben alle ein 
Gravitationszentrum, um welches sich alles dreht. Im Zentrum der Galaxie befindet sich ein Schwarzes 
Loch welches umgeben von einem sogenannten Bulge ist. Der Bulge ist eine riesige Sternenansamung auf relativ engem Raum. Um das 
Zentrum herum erstreckt sich eine galaktische scheibe, in der sich die meisten Sterne, interstellarem Gas und Staub befinden. 
Je nach Galaxieart sind die "Arme" anders angeordnet, z.B. sind diese spriralförmig in der Spiralgalaxie. Um die Scheibe herum befindet 
sich ein spärisches Halo, in dem sich Sterne und Kugelsternhaufen befinden. 
(Ignasi Ribas, Chris Hadfield, 2021, S. 20-24)

\subsection{Rotationskurve}

Rotationskurven geben an, wie sich die Geschwindigkeit von Objekten innerhalb einer Galaxie verändern im Vergleich zum Abstand vom Zentrum. 
Damit die Rotationskurve und die Massenverteilung von Galaxien berechnet werden kann, muss ihre Struktur sowie ihre Zusammensetzung 
aus Sternen, Staub, stellarem Gas und dunkler Materie bekannt sein. (Ann-Kristin Möller,2010, S.5) 

Die universale Rotationskurve, ist eine mathematische Formel, die die Rotationskurven von Galaxien anhand ihrer Gesamthelligkeit und ihres Radius charakterisiert.
Wodurch darauf geschlossen werden kann, dass die hellsten Galaxien eine leicht Abfallende Rotationskurve haben, mittlere eine constant flache und dunklere 
Galaxien eine monoton ansteigende Rotationskurve haben. (Yoshiaki Sofue and Vera Rubin,2000, S.7 und s 17 )

\subsection{Galaxien drehen sich zu schnell}

Die Formel zu Berechnung von der Orbital Geschwindigkeit lautet:
\begin{align*}    
    v = \sqrt[2]{\frac{G \cdot M}{r}}
\end{align*}
Wobei G die Gravitstionskonstante ist, M ist die Masse des Zentralgestirns und r die distanz zwischen den Objekten. Bei M muss 
beachtet werden, das die gesamte Masse innerhalb der jeweiligen Planetenbahn zählt. 
Von dieser Formel würde man darauf schliessen, je weiter weg das Objekt desto lamgsamer bewegt sich 
dieses, da durch r dividiert wird, was auch bei den Planeten in unserem Sonnensystem zu trifft. Dieses Gesetz kann auf alle
Systeme wo ein Körper ein Zentralbereich umkreist angewendet werden.(Thomas Bührke, 2022, S. 21)
Ähnlich ist es bei den Galaxien, jedoch befindet sich im Zentrum nicht eine einzige dominierende Masse sondern den sogenannten Bulge. Dort befnden 
sich soviele Sterne, dass sie nicht mehr einzeln zu erkenen sind. Mit der Spektroskopie der fernen Sternsysteme können Aufschlüsse über die Bewegung und Masse der Sterne
gezogen werden. Jedoch erweisst sich die Aufnahme von Spektren als Herausforderung, vor allem die Bestimmung der Geschwindigkeit von Objekten 
in den lichtschwachen aussenberichen der Scheibe. Vergleicht man die hellen Zentralgebieten mit den Äusseren sind dort nicht mehr viele Sterne zu erwarten.
Dadurch fasste man den Schluss, das die Geschwindigkeit mit wachsendem Abstand zum Zentrum abnehemen muss, gleich wie im Sonnensystem. 
Jedoch beobachtete Horace Babcock, der mit einem neuem Spektrograf die Rotationskurve der Andromedar-Galaxie mass, etwas Überaschendes. Anstelle einer abfallenden Rotationskurve
blieb sie bis in die Aussenbereiche konstant, sie schien sogar eher zuzunehmen. Wenn die Geschwindigkeit der Körper mit Wachsendem abstand vom Zentrum nicht abnehemen, sondern konstant bleiben, 
muss sich innerhalb der Umlaufbahn unsichtbare Materie befinden, welche eine erhebliche Schwerkraft ausübt. Vereinfacht: Galaxien haben zu wenig Masse als das sie sich so schnell drehen könnten.
Für ein besseres verständiss kann man sich ein Kettenkaruesl vorstellen, dessen Sessel sich so schnell drehen das die Ketten reissen. Jedoch halten die Spieralgalaxien aus unnerklärlichen Gründen zusammen. (Thomas Bührke,2022, S. 26)

\subsubsection{mögliche Begründungen}
Der Astronom Jan Hendrik Oort stiess auf das Gleiche Problem, als er Daten von der Galaxie NGC 3115 auswertete. Diese Galaxie zälht zu den Linsenförmigen, mit einem zentralen Bulge und einer ausgeprägten Scheibe ohne Spiralarme. 
Gleich wie Babcock in der Andromedar Galaxie zu hohe Geschwindigkeiten der Objekte weit von Zentrum entfert mas, kam Oort auf dasselbe: Die Umlaufgeschwindigkeiten der Sterne nimmt mit wachsendem Abstand von Zentrum nicht ab.
Oort kam zum schluss, dass die Massenverteilung fast keine Beziehung zu der des Lichts habe. Er vermutete, dass es sehr lichtschwache und damit nichtmehr erkennbare Zwergesterne hat welche der Grund für die Fehlende Masse war.
Vera Rubin eine der ersten Frauen die sich in dieser zu der zeit Männerdominierten Forschungsrichtung durchsetzte, beobachtete die Geschwindikeit von etwa 1000 Sternen in der Milchstrasse und kam zu demselben schluss: Die Rotationskurve 
ist flach und nimmt nicht ab, wie man es erwarten würde.  (Thomas Bührke,2022, S. 29)





