%!TEX root = ../maturaarbeit.tex

\section{Grundlagen}

\subsection{Aufbau von Galaxien}

Es gibt verschiede Arten von Galaxien wie zum Beispiel die Eliptische-, Spiral-, Balken-, Linsenförmige-, 
und die Irreguläre Galxien. Sie unterscheiden sich in Struktur und Form. Grundsätzlich haben alle ein Gravitationszentrum, um welches sich alles 
dreht.     



\subsection{Galxien drehen sich zu schnell}

Die Formel zu Berechnung von der Orbital Geschwindigkeit lautet:
\begin{align*}    
    v = \sqrt[2]{\frac{G * M}{r}}
\end{align*}

Wobei G die Gravitstionskonstante ist, M ist die Masse des Zentralgestirns und r die distanz zwischen den Objekten. Bei M muss 
beachtet werden, das die gesamte Masse innerhalb der jeweiligen Planetenbahn zählt. 
Von dieser Formel würde man darauf schliessen, je weiter weg das Objekt desto lamgsamer bewegt sich 
dieses, da durch r dividiert wird, was auch bei den Planeten in unserem Sonnensystem zu trifft. Dieses Gesetz kann auf alle
Systeme wo ein Körper ein Zentralbereich umkreist angewendet werden.  
\subsection{Rotationskurve}


\subsection{Dunkle Materie}

Es gibt zwei Arten von Dunkler Materie. Die baryonische und die nicht baryonische Dunkle
Materie. Unter baryonischer Materie versteht man "normale" Materie welche aus Elektronen, Neutronen 
und Protonen besteht. Damit sind zum Beispiel Objekte wie Massenarme und daher Leuchtschwache Sterne
gemeint. Diese haben die Fachbezeichnung "MACHO" was für MAssive Compact Halo Objects steht. Diese können
nicht Direkt beobachtet werden. Wodruch sie wortwörlich einfach dunkle Materie sind.
Nicht baryonische Materie, wie der Name verrät besteht nicht aus Elektronen, Neutronen und Protonen. Es gibt
verschiede Vermutungen aus was diese bestehen. (Sibylle Anderl,2023, S. 51) 




\subsection{Erstes Unterkapitel des ersten Kapitels}


\subsection{Zweites Unterkapitel des ersten Kapitels}

\lipsum[1]

Füge mit \textit{newpage} einen Seitenumbruch ein.

