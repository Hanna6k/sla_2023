%!TEX root = ../maturaarbeit.tex

\section{Schlusswort}

Die Untersuchung der Rotationskurven von Galaxien hat gezeigt, dass die beobachteten Geschwindigkeiten der Sterne und der Gase in den äusseren Regionen von Galaxien oft nicht den Erwartungen entsprechen, die auf Grundlage der sichtbaren Materie allein getroffen wurden. Diese Diskrepanz zwischen den beobachteten Rotationskurven und den theoretisch vorhergesagten Kurven kann nur durch die Anwesenheit von Dunkler Materie erklärt werden.
Dunkle Materie, obwohl unsichtbar und noch nicht bewiesen, übt eine starke Gravitationskraft aus, die die sichtbare Materie in einer Galaxie beeinflusst. Die Dunkle Materie bildet eine unsichtbare Halo-Struktur um das Zentrum der Galaxie herum und trägt wesentlich zur Rotationsdynamik bei. Die Anwesenheit von Dunkler Materie erklärt, warum die Rotationskurven in den äusseren Regionen von Galaxien flach bleiben, anstatt wie erwartet, abzufallen.

Obwohl Dunkle Materie noch nie nachgewiesen wurde, deutet sehr viel darauf hin, dass es sie gibt. Mich fasziniert, wie durch Beobachten von Sternen, die über Tausende von Lichtjahren entfernt sind, so viel herausgefunden werden kann.
Der Gedanke, dass Dunkle Materie ein so wichtiger Teil sein muss, finde ich erstaunlich.
 







