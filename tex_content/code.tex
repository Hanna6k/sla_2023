%!TEX root = ../maturaarbeit.tex

\section{Code}

\subsection{Python Code}

Code-Ausschnitt:

\begin{python}
import numpy as np

print("Hello World")

for i in range(3):
	print(i)
\end{python}

Code direkt im Fliesstext integriert: \pythoninline{print(42)}.

\subsection{JavaScript and TypeScript Code}

\begin{javascript}
function greeter(person: string) {
	return "Hello, " + person;
} 

let user = "Jane User";

document.body.textContent = greeter(user);
\end{javascript}

Code direkt im Fliesstext integriert: \javascriptinline{let user = "Jane User";}.


\subsection{C\# Code}

\begin{csharp}
using System;

namespace HelloWorld
{
  class Program
  {
    static void Main(string[] args)
    {
      Console.WriteLine("Hello World!");    
    }
  }
}
\end{csharp}

Code direkt im Fliesstext integriert: \csharpinline{Console.WriteLine("Hello World!");}.