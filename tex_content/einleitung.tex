%!TEX root = ../maturaarbeit.tex

\section{Einleitung}

Der Nachthimmel fasziniert Menschen schon seit Urzeiten. Die Ägypter nutzten Sterne für die Zeitmessung, \cite{Hornung1998} die Babylonier entwickelten komplexe astronomische Kalender, und die antiken Griechen betrachteten die Bewegungen der Himmelskörper als göttliche Botschaften \cite{Friedrich1968}.
Das geozentrische Weltbild, bei dem sich alles um die Erde dreht, wurde durch das heliozentrische Weltbild, bei dem sich alles um die Sonne dreht, ersetzt \cite{Walker1999}. Danach folgte die Überlegung, dass sich die Sonne mit ihren Planeten um ein schwarzes Loch mit Millionen von anderen Sternen drehen würde. 
Aus der ursprünglichen Idee einer einzigen Galaxie entwickelte sich das Wissen, dass es möglicherweise unendlich viele Galaxien gibt. Die Annahme, dass das Universum statisch sei, konnte durch Beweise widerlegt werden und mit der beschleunigten Expansion des Universums ersetzt werden \cite{Bührke2022}. 
Durch Beobachtungen, Messungen und Überlegungen versuchen Astronom/innen unseren Nachthimmel zu verstehen und erklären zu können. 

Seitdem ich einen Kurs über Zeitreisen und Schwarze Löcher besucht habe, in welchem Dunkle Materie thematisiert worden ist, bin ich fasziniert von diesem Thema. Aus diesem Grund habe ich mich für das Thema der Dunklen Materie entschieden. Die zentrale Fragestellung der Arbeit lautet: Wie beeinflusst Dunkle Materie die Rotationskurve von Galaxien?
Um dies zu erörtern, behandle ich folgende Themen:
In einem ersten Teil beschreibe ich die vier fundamentalen Wechselwirkungen und anschliessend was Dunkle Materie sein könnte. Im zweiten Teil werde ich als Erstes den Aufbau von Galaxien beschreiben, danach erkläre ich, was die Rotationskurve ist. Abschliessend erläutere ich den Zusammenhang zwischen der Rotationskurve von Galaxien und wie Dunkle Materie diese begründen kann.
