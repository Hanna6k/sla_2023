%!TEX root = ../maturaarbeit.tex

\section{Einleitung}

Der Nachhimmel faszinierte Menschen schon seit Urzeiten. Die Ägypter nutzen Sterne für die Zeitmessung, die Babylonier entwickelten komplexe astronomische
Kalender, und die antiken Griechen betrachteten die Bewegungen der Himmelskörper als göttliche Botschaften. Bis hin zum geozentrischem Weltbild welches dann vom 
heliozentrischen ersetz wurde. Vom Gedanke alles dreht sich um die Sonne bis hin zur Überlegung, die Sonne bewegt sich mit ihren Platen um ein schwarzes Loch mit millionen von anderen Sternen. Von der Idee es gäbe nur eine 
einzige Glaxie bis hin zum wissen es gibt möglicherweise unendlich viele Galaxien. Von der Annahme das das Universum sei statisch bis hin zum Beweis der beschleunigten Expansion des Universums. Durch 
Beobachtungen und Messungen versuchen Astronomen unser Nachhimmel zu verstehen und erklären zu können.


In einem ersten Teil werde ich  auf Galaxien und dessen eigenschaften eingehen. Im zweiten Teil werde ich als erstes die vier fundamentale Wechselwirkungen
erklären, dannach was dunkle Materie sein könnte und warum es sie geben muss. Anschliessend erläutere ich den Zusammenhang zwischen der Rotationskurve von 
Galaxien und wie dunkle Materie diese ergründen kann.

Damit beantworte ich folgende Fragestellung: Wie beeinflusst dunkle Materie die Rotationskurve von Galaxien?

