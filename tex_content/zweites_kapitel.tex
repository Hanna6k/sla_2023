%!TEX root = ../maturaarbeit.tex

\section{Zweites Kapitel}

\lipsum[4]

\subsection{Bilder einfügen}

Füge ein Bild ein:
\begin{figure}[H]
	\centering
	\includegraphics[width=0.4\textwidth]{figures/grumpy_cat}
	\caption[Eintrag in Abbildungsverzeichnis von Grumpy Cat]{Grumpy cat. The Grumpy Cat is a grumpy cat.}
	\label{fig grumpy cat.}
\end{figure}
Natürlich kann man auch Referenzen auf Bilder einfügen.
Bild \ref{fig grumpy cat.} zeigt die Grumpy Cat.
Die Breite/Höhe eines Bildes kann man angeben z.B. mit width=3cm oder height=15mm.
Soll das Bild die halbe Breite des Textes haben, kann man angeben
width=0.5\textbackslash textwidth

\newpage

\subsection{Listen}

Liste
\begin{itemize}
	\item Erstes Element
	\item Zweites Element
	\item Drittes Element
	\item Viertes Element
\end{itemize}

Sehr kompakte Liste:
\begin{itemize}
	\vspace{-\topsep}
	\setlength{\itemsep}{0pt}\setlength{\parskip}{0pt}
	\item Erstes Element
	\item Zweites Element
	\item Drittes Element
	\item Viertes Element
\end{itemize}

\subsection{Aufzählung}

Aufzählung mit Zahlen nummeriert:
\begin{enumerate}
	\item Erstes Element
	\item Zweites Element
	\item Drittes Element
	\item Viertes Element
\end{enumerate}

Aufzählung mit Grossbuchstaben nummeriert:
\begin{enumerate}[label=\Alph*]
	\item Erstes Element
	\item Zweites Element
	\item Drittes Element
	\item Viertes Element
\end{enumerate}

Aufzählung mit krossbuchstaben nummeriert:
\begin{enumerate}[label=\alph*)]
	\item Erstes Element
	\item Zweites Element
	\item Drittes Element
	\item Viertes Element
\end{enumerate}

Aufzählung mit römischen Buchstaben nummeriert:
\begin{enumerate}[label=(\roman*)]
	\item Erstes Element
	\item Zweites Element
	\item Drittes Element
	\item Viertes Element
\end{enumerate}

\subsection*{Kapitel ohne Nummerierung}

Wer meine Nummer findet, soll sich bitte bei mir melden!

\subsection{Hilfreiche Links}

\subsubsection{Tabelle}

Verwende den folgenden Tabellen-Generator, um einfache Tabellen zu erzeugen: \url{https://www.tablesgenerator.com}

\subsubsection{Symbole}

Falls du nicht weisst, wie man ein gewisses Symbol in LaTeX darstellst, so hilft dir folgende Seite, um dieses zu identifizieren: \url{http://detexify.kirelabs.org/classify.html}

% \lipsum[1]

\subsection{Zweites Unterkapitel des zweiten Kapitels}

\lipsum[2-3]
